% -------------------------------------------------------
% File: povm_examples.tex
% Supplementary POVM and measurement-instrument examples
% for the Quantum–Reflexive Control Framework.
% -------------------------------------------------------

\section{Supplementary POVM and Measurement Instrument Examples}

This document provides an extended catalogue of POVM constructions used in the
Quantum–Reflexive Control Framework, including soft measurements, sharp projective
measurements, asymmetric POVMs, informationally complete POVMs, and the associated
Kraus-operator instruments used in the prophecy measurement $M$.

All Hilbert spaces are finite-dimensional, and all POVM elements $\{E_i\}$ satisfy
\[
E_i \succeq 0,
\qquad
\sum_i E_i = \id.
\]

% -------------------------------------------------------
\subsection{Binary Soft Measurement with Sharpness Parameter $\alpha$}

The primary POVM used in the main text is the binary soft measurement on a qubit:

\[
E_0 = (1-\alpha)\frac{\id}{2} + \alpha \ket{0}\!\bra{0},
\qquad
E_1 = (1-\alpha)\frac{\id}{2} + \alpha \ket{1}\!\bra{1},
\]

with $\alpha \in [0,1]$ determining the measurement sharpness.

\paragraph{Special cases.}
\begin{itemize}
\item $\alpha = 0 \implies E_0 = E_1 = \frac{\id}{2}$ (no information).
\item $\alpha = 1 \implies E_0 = \ket{0}\!\bra{0},\ E_1 = \ket{1}\!\bra{1}$ (projective measurement).
\end{itemize}

\paragraph{Associated Kraus operators.}
We take the canonical instrument:
\[
M_i = \sqrt{E_i}.
\]

Since $E_0$ and $E_1$ are diagonal, so are their square roots:
\[
M_0 =
\begin{pmatrix}
\sqrt{\tfrac{1+\alpha}{2}} & 0 \\[2pt]
0 & \sqrt{\tfrac{1-\alpha}{2}}
\end{pmatrix},
\qquad
M_1 =
\begin{pmatrix}
\sqrt{\tfrac{1-\alpha}{2}} & 0 \\[2pt]
0 & \sqrt{\tfrac{1+\alpha}{2}}
\end{pmatrix}.
\]

The corresponding measurement instrument is:
\[
\mathcal{M}(\rho) = \sum_{i=0}^{1} M_i \rho M_i^\dagger.
\]

% -------------------------------------------------------
\subsection{Asymmetric Binary POVM}

In scenarios where one outcome carries more predictive weight, we use:

\[
E_0 = a\,\ket{0}\!\bra{0} + b\,\ket{1}\!\bra{1},
\qquad
E_1 = (1-a)\ket{0}\!\bra{0} + (1-b)\ket{1}\!\bra{1},
\]

with parameters $a,b \in [0,1]$.

\paragraph{Normalization.}
\[
E_0 + E_1 = (a + 1 - a)\ket{0}\bra{0} + (b + 1 - b)\ket{1}\bra{1} = \id.
\]

\paragraph{Kraus operators.}
\[
M_0 =
\begin{pmatrix}
\sqrt{a} & 0\\
0 & \sqrt{b}
\end{pmatrix},
\quad
M_1 =
\begin{pmatrix}
\sqrt{1-a} & 0\\
0 & \sqrt{1-b}
\end{pmatrix}.
\]

This POVM allows control over measurement bias and can model asymmetric prophetic pressure.

% -------------------------------------------------------
\subsection{Three-Outcome POVM (Trine Measurement)}

For richer measurement structure on a qubit, the trine POVM is useful:

\[
E_k = \frac{2}{3}\ket{\psi_k}\!\bra{\psi_k},
\qquad
k \in \{0,1,2\},
\]

with states spaced $120^\circ$ apart on the Bloch sphere:
\[
\ket{\psi_k}
= \frac{1}{\sqrt{2}}
\left(
\ket{0} + e^{2\pi i k / 3}\ket{1}
\right).
\]

The POVM satisfies:
\[
\sum_{k=0}^2 E_k = \id.
\]

\paragraph{Kraus operators.}
\[
M_k = \sqrt{\frac{2}{3}}\,\ket{\psi_k}\!\bra{\psi_k}.
\]

This POVM is informationally rich and useful when the regulator wants to distribute information across several possible outputs.

% -------------------------------------------------------
\subsection{Four-Outcome POVM (SIC-POVM)}

A symmetric informationally complete POVM (SIC-POVM) on a qubit has elements:

\[
E_k = \frac{1}{4}\ket{\phi_k}\!\bra{\phi_k},
\quad k=0,1,2,3,
\]

where the $\ket{\phi_k}$ form a regular tetrahedron on the Bloch sphere.

Explicitly:
\[
\ket{\phi_0} = \ket{0},\qquad
\ket{\phi_k} =
\frac{1}{\sqrt{3}}
\left(
\ket{0} + \sqrt{2} e^{2\pi i (k-1)/3}\ket{1}
\right).
\]

Normalization:
\[
\sum_{k=0}^3 E_k = \id.
\]

\paragraph{Kraus operators.}
\[
M_k = \frac{1}{2}\ket{\phi_k}\!\bra{\phi_k}.
\]

This POVM is informationally complete and can fully reconstruct the quantum state, making it useful for high-information prophecy.

% -------------------------------------------------------
\subsection{Projective Measurements (PVMs)}

A sharp projective measurement on a qubit in any basis $\{\ket{v_0},\ket{v_1}\}$ is:

\[
E_0 = \ket{v_0}\!\bra{v_0},
\qquad
E_1 = \ket{v_1}\!\bra{v_1},
\qquad
\ket{v_1} \perp \ket{v_0}.
\]

Instrument has Kraus operators:
\[
M_0 = \ket{v_0}\!\bra{v_0},\qquad
M_1 = \ket{v_1}\!\bra{v_1}.
\]

These are extremal POVMs (maximal sharpness).

% -------------------------------------------------------
\subsection{Effects on the Prophecy–Response Pipeline}

Given a POVM $\{E_i\}$ and Kraus operators $\{M_i\}$, the prophecy measurement part of
$\Phi^{\mathcal{Q}}$ acts as:

\[
M(\rho) = \sum_{i} (\id_X \otimes M_i)\,\rho\,(\id_X \otimes M_i^\dagger).
\]

Key observations:

\begin{itemize}
\item Soft POVMs ($\alpha < 1$) reduce disturbance and leakage.
\item Sharp POVMs ($\alpha = 1$) maximize information but also disturbance.
\item Multi-outcome POVMs distribute information across outputs, useful for regulators.
\item SIC-POVMs maximize information per outcome; regulators may need noise injection.
\end{itemize}

% -------------------------------------------------------
\subsection{Implementation Notes}

\begin{itemize}
\item All examples here are numerically stable and used in the project’s simulation engine.
\item Multi-outcome POVMs extend naturally to the bipartite space via $\id_X \otimes M_i$.
\item Any POVM can be converted into a measurement instrument via standard Kraus dilation.
\end{itemize}

This file completes the supplementary POVM catalogue used in the Quantum–Reflexive Control Framework.
