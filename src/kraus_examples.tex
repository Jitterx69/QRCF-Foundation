% -------------------------------------------------------
% File: kraus_examples.tex
% Supplementary examples of Kraus operators for channels
% used in the Quantum–Reflexive Control Framework.
% -------------------------------------------------------

\section{Supplementary Kraus Operator Examples}

This document provides concrete examples of Kraus operator constructions used throughout
the Quantum–Reflexive Control Framework. Each example is fully specified in matrix form and
intended for inclusion in simulations or as teaching illustrations in the main text.

\subsection{Amplitude-Damping Channel on a Single Qubit}

One of the central world-evolution channels considered in the framework is the amplitude-damping
channel with parameter $\gamma \in [0,1]$. Its Kraus operators are:

\[
K_0 =
\begin{pmatrix}
1 & 0 \\
0 & \sqrt{1-\gamma}
\end{pmatrix},
\qquad
K_1 =
\begin{pmatrix}
0 & \sqrt{\gamma} \\
0 & 0
\end{pmatrix}.
\]

These satisfy:

\[
K_0^\dagger K_0 + K_1^\dagger K_1 = \id.
\]

They model irreversible decay, and in the prophecy–response operator $\Phi^{\mathcal{Q}}$, they
encode the natural ``world dynamics'' applied after measurement, regulation, and agent action.

\subsection{Phase-Damping Channel}

The phase-damping (dephasing) channel is useful in alternative implementations of $\Phi^{\mathcal{Q}}$
when modelling uncertainty or decoherence. Its Kraus operators are:

\[
L_0 = \sqrt{1-p}\,\id,
\qquad
L_1 =
\sqrt{p}
\begin{pmatrix}
1 & 0 \\
0 & 0
\end{pmatrix},
\qquad
L_2 =
\sqrt{p}
\begin{pmatrix}
0 & 0 \\
0 & 1
\end{pmatrix}.
\]

These yield a CPTP channel satisfying
\[
\sum_i L_i^\dagger L_i = \id.
\]

\subsection{Two-Qubit Extension}

When acting on a bipartite system $X \otimes V$, a single-qubit Kraus operator $K_j$ extends to
the full Hilbert space as:

\[
\widetilde{K}_j = K_j \otimes \id_V.
\]

This is used in the definition of the world-evolution map $\mathcal{E}$:

\[
\mathcal{E}(\rho) = \sum_j (K_j \otimes \id_V)\,\rho\,(K_j^\dagger \otimes \id_V).
\]

\subsection{Depolarizing Channel (Regulator Example)}

The regulator uses a depolarizing channel on subsystem $V$ of dimension $d=2$ (for the toy
model). Its Kraus form is:

\[
D_0 = \sqrt{1-\lambda}\,\id, \qquad
D_1 = \sqrt{\frac{\lambda}{3}}\,X, \qquad
D_2 = \sqrt{\frac{\lambda}{3}}\,Y, \qquad
D_3 = \sqrt{\frac{\lambda}{3}}\,Z,
\]

where $X,Y,Z$ are Pauli operators.

When extended to two qubits:

\[
\widetilde{D}_k = \id_X \otimes D_k.
\]

\subsection{General Notes for Implementation}

\begin{itemize}
    \item All channels must satisfy the completeness relation:
    \[
    \sum_i K_i^\dagger K_i = \id.
    \]
    \item Composition of channels corresponds to multiplication of Kraus sets:
    if $\mathcal{E}_1$ has Kraus $\{A_i\}$ and $\mathcal{E}_2$ has $\{B_j\}$, then
    \[
    (\mathcal{E}_2 \circ \mathcal{E}_1)(\rho)
    = \sum_{i,j} B_j A_i\,\rho\,A_i^\dagger B_j^\dagger.
    \]
    \item All numerical simulations use the matrix forms listed here.
\end{itemize}

This completes the set of supplementary Kraus-operator examples for the project.
