% -------------------------------------------------------
% File: operator_appendix.tex
% Operator-theoretic expansions for the
% Quantum–Reflexive Control Framework.
% -------------------------------------------------------

\section{Additional Operator-Theoretic Expansions}

This appendix develops supplementary operator-theoretic results used throughout the analysis of
the prophecy--response operator $\Phi^{\mathcal{Q}}$. We provide explicit block-matrix forms,
composition rules, continuity theorems, dilation constructions, and operator inequalities that were
not included in the main text.

Throughout, $\mathcal{H}_X$ and $\mathcal{H}_V$ denote the latent and prophecy Hilbert spaces,
and the full system is $\mathcal{H} = \mathcal{H}_X \otimes \mathcal{H}_V$ with
$\dim(\mathcal{H}_X)=d_X$ and $\dim(\mathcal{H}_V)=d_V$.

% -------------------------------------------------------
\subsection{Block-Operator Representation of Measurement Instruments}

Let $\{M_i\}$ denote the Kraus operators for the prophecy measurement acting on $\mathcal{H}_V$.
Define:
\[
\widetilde{M}_i = \id_X \otimes M_i.
\]

Any operator $\rho \in \mathcal{B}(\mathcal{H})$ can be written in block form:
\[
\rho =
\begin{pmatrix}
\rho_{00} & \rho_{01} & \cdots \\
\rho_{10} & \rho_{11} & \cdots \\
\vdots & \vdots & \ddots
\end{pmatrix},
\]
where each block is a $d_V \times d_V$ matrix.

Then:
\[
M(\rho)
=
\sum_i \widetilde{M}_i \rho \widetilde{M}_i^\dagger
=
\sum_i
\begin{pmatrix}
M_i \rho_{00} M_i^\dagger & M_i\rho_{01}M_i^\dagger & \cdots \\
M_i \rho_{10} M_i^\dagger & M_i\rho_{11}M_i^\dagger & \cdots \\
\vdots & \vdots & \ddots
\end{pmatrix}.
\]

\paragraph{Observation.}
The measurement channel acts \emph{independently} on each $X$-indexed block.

This structure is crucial for analyzing partial decoherence and leakage control.

% -------------------------------------------------------
\subsection{Block Form of the Regulator Channel}

The regulator acts only on $V$, so writing $\sigma$ in blocks as above:
\[
\rho_{\mathrm{rel}}(\sigma)
= (1-\lambda)\sigma
 + \lambda\,(\id_V/ d_V) \otimes \Tr_V(\sigma).
\]

In block form:
\[
\Tr_V(\sigma)
=
\begin{pmatrix}
\Tr(\sigma_{00}) & \Tr(\sigma_{01}) & \cdots \\
\Tr(\sigma_{10}) & \Tr(\sigma_{11}) & \cdots \\
\vdots & \vdots & \ddots
\end{pmatrix},
\]

and thus:
\[
\rho_{\mathrm{rel}}(\sigma)
=
(1-\lambda)
\begin{pmatrix}
\sigma_{00} & \sigma_{01} & \cdots \\
\sigma_{10} & \sigma_{11} & \cdots \\
\vdots & \vdots & \ddots
\end{pmatrix}
+
\lambda
\begin{pmatrix}
\Tr(\sigma_{00})\frac{\id_V}{d_V} & \Tr(\sigma_{01})\frac{\id_V}{d_V} & \cdots \\
\Tr(\sigma_{10})\frac{\id_V}{d_V} & \Tr(\sigma_{11})\frac{\id_V}{d_V} & \cdots \\
\vdots & \vdots & \ddots
\end{pmatrix}.
\]

This clarifies how regulator noise influences cross-block coherence.

% -------------------------------------------------------
\subsection{Block Form of Agent Response}

Let $U_i$ act on $\mathcal{H}_X$ only. Then:
\[
\pi(\sigma)
=
\sum_i (U_i \otimes \id_V)
      \, \sigma_i \,
      (U_i^\dagger \otimes \id_V),
\]
where $\sigma_i = (\id_X \otimes M_i)\sigma(\id_X \otimes M_i^\dagger)$.

Blockwise:
\[
\sigma_i =
\begin{pmatrix}
M_i\sigma_{00}M_i^\dagger & M_i\sigma_{01}M_i^\dagger & \cdots \\
M_i\sigma_{10}M_i^\dagger & M_i\sigma_{11}M_i^\dagger & \cdots \\
\vdots & \vdots & \ddots
\end{pmatrix}.
\]

Applying $U_i$ conjugation permutes or rotates blocks depending on how $U_i$ acts on $X$.

This formalism extends all agent-response policies, including deterministic, stochastic, and mixed-unitary responses.

% -------------------------------------------------------
\subsection{Composition Structure of the Full Operator}

The prophecy–response operator is:

\[
\Phi^{\mathcal{Q}}
= \mathcal{E} \circ \pi \circ \rho_{\mathrm{rel}} \circ M.
\]

In block form:

\[
\Phi^{\mathcal{Q}}(\rho)
=
\sum_{i,j,k} (K_j \otimes \id_V)
             (U_i \otimes \id_V)
             \widetilde{M}_k
             \rho
             \widetilde{M}_k^\dagger
             (U_i^\dagger \otimes \id_V)
             (K_j^\dagger \otimes \id_V).
\]

Thus $\Phi^{\mathcal{Q}}$ always admits Kraus operators:
\[
A_{ijk} = (K_j U_i) \otimes M_k.
\]

\begin{proposition}[Explicit Kraus Form of $\Phi^{\mathcal{Q}}$]
\[
\Phi^{\mathcal{Q}}(\rho)
= \sum_{ijk} A_{ijk}\rho A_{ijk}^\dagger.
\]
\end{proposition}

\begin{proof}
Direct expansion of the composed CPTP maps.
\end{proof}

This is the most explicit possible representation and is used for spectral and contraction analysis.

% -------------------------------------------------------
\subsection{Continuity of $\Phi^{\mathcal{Q}}$}

\begin{theorem}[Lipschitz Continuity]
There exists constant $L \le 1$ such that
\[
\|\Phi^{\mathcal{Q}}(\rho) - \Phi^{\mathcal{Q}}(\sigma)\|_1
\le L \|\rho - \sigma\|_1.
\]
\end{theorem}

\begin{proof}
Each component map is either non-expansive or strictly contractive.  
Specifically:
\[
\|M(\rho)-M(\sigma)\|_1 \le \|\rho-\sigma\|_1,
\]
\[
\|\rho_{\mathrm{rel}}(A)-\rho_{\mathrm{rel}}(B)\|_1 \le (1-\lambda)\|A-B\|_1,
\]
\[
\|\pi(A)-\pi(B)\|_1 \le \|A-B\|_1,
\]
\[
\|\mathcal{E}(A)-\mathcal{E}(B)\|_1 \le \|A-B\|_1.
\]
Multiply bounds.
\end{proof}

% -------------------------------------------------------
\subsection{Differentiability and Fréchet Derivatives}

We define the Fréchet derivative of $\Phi^{\mathcal{Q}}$ at $\rho$ in direction $H$:

\[
D\Phi^{\mathcal{Q}}(\rho)[H]
=
\sum_{ijk}
A_{ijk} H A_{ijk}^\dagger.
\]

\begin{proposition}
$D\Phi^{\mathcal{Q}}(\rho)$ is a completely positive linear map.
\end{proposition}

\begin{proof}
Follows from Kraus form above.
\end{proof}

\begin{corollary}
\[
\|D\Phi^{\mathcal{Q}}(\rho)\|_{1 \to 1} \le L \le 1.
\]
\end{corollary}

Hence $\Phi^{\mathcal{Q}}$ is globally Lipschitz differentiable.

% -------------------------------------------------------
\subsection{Spectral Properties}

\begin{theorem}[Spectrum Inside the Unit Disk]
All eigenvalues of $\Phi^{\mathcal{Q}}$ satisfy $|\lambda| \le 1$.
\end{theorem}

\begin{proof}
As a trace-non-increasing CP map on a finite-dimensional space, its operator norm satisfies
$\|\Phi^{\mathcal{Q}}\|_{1\to1} \le 1$, hence its spectral radius is $\le 1$.
\end{proof}

\begin{corollary}
If $\lambda>0$ (depolarization), then the spectral radius is $<1$, guaranteeing contraction.
\end{corollary}

% -------------------------------------------------------
\subsection{Characterization of Fixed Points}

\begin{theorem}[Fixed-Point Condition]
$\rho^\ast$ is a fixed point of $\Phi^{\mathcal{Q}}$ iff
\[
\rho^\ast = \sum_{ijk} A_{ijk} \rho^\ast A_{ijk}^\dagger.
\]
\end{theorem}

\begin{proposition}[Fixed-Point Subspace]
The set of fixed points is a convex linear subspace of $\mathcal{B}(\mathcal{H})$.
\end{proposition}

\begin{proof}
Follows from linearity and positivity of $\Phi^{\mathcal{Q}}$.
\end{proof}

In contractive regimes, the fixed-point subspace collapses to a unique point.

% -------------------------------------------------------
\subsection{Additional Notes on Operator Ordering and Composition}

\paragraph{Important identity.}
If
\[
\Phi = \sum_i A_i (\cdot) A_i^\dagger,\qquad
\Psi = \sum_j B_j (\cdot) B_j^\dagger,
\]
then their composition satisfies:
\[
\Psi\circ\Phi = \sum_{i,j} (B_j A_i) (\cdot) (A_i^\dagger B_j^\dagger).
\]

This gives the exact structure of iteration:
\[
\Phi^{t}(\rho) = \sum_{i_1,\ldots,i_t}
A_{i_t}\cdots A_{i_1}\,\rho\,A_{i_1}^\dagger \cdots A_{i_t}^\dagger.
\]

\paragraph{Implication.}
Iterating the prophecy–response operator builds a “path-sum” over all possible measurement and
response histories.

% -------------------------------------------------------
\subsection{Summary}

This appendix provides:

\begin{itemize}
    \item block matrix forms of measurement, regulation, and agent maps,
    \item explicit Kraus form for the full operator $\Phi^{\mathcal{Q}}$,
    \item continuity, differentiability, and Lipschitz bounds,
    \item spectral characterizations,
    \item structural results for fixed points.
\end{itemize}

These tools support deeper mathematical analysis and facilitate efficient numerical implementation.
