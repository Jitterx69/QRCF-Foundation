% -------------------------------------------------------
% SECTION 5: ETHICAL CONTROL THEORY
% -------------------------------------------------------

Quantum prophetic systems present a unique regulatory challenge: prophecy releases information
that may perturb the world in harmful ways, yet restricting prophetic information may reduce the
system’s ability to decrease future harm. Ethical control requires a principled method to balance
information disclosure, fairness constraints, privacy, robustness, and the minimization of expected
societal harm. In this section we formalize these considerations.

\subsection{Harm Functional}

Let $\mathcal{H}_X$ contain a designated ``crash'' state $\ket{1}$ representing catastrophic system
failure. The probability of crash under a state $\rho$ is
\[
    P_{\mathrm{crash}}(\rho)
    =
    \Tr\!\left[
        (\dyad{1}_X \otimes \id_V) \rho
    \right].
\]

\begin{definition}[Harm Functional]
Let $c_1, c_2 > 0$. The harm associated with a density operator $\rho$ is
\[
    H(\rho)
    =
    c_1 P_{\mathrm{crash}}(\rho)
    +
    c_2\, D(\rho),
\]
where $D(\rho)$ is the measurement-induced disturbance defined by
\[
    D(\rho) = \| \rho - M(\rho) \|_1.
\]
\end{definition}

\begin{remark}
The disturbance term penalizes excessive measurement sharpness, reflecting the trade-off between
information gain and physical perturbation of the quantum state.
\end{remark}

\subsection{Information Leakage Constraint}

The ethical regulator aims to limit the amount of information transferred from the latent state to
released prophetic signals.

\begin{definition}[Quantum Mutual Information]
For a bipartite state $\rho_{XY}$,
\[
    I(X;Y)
    =
    S(\rho_X)
    +
    S(\rho_Y)
    -
    S(\rho_{XY}),
\]
with marginals $\rho_X = \Tr_Y(\rho)$ and $\rho_Y = \Tr_X(\rho)$.
\end{definition}

Applying this to prophecy:

\begin{definition}[Leakage Functional]
Let $\rho_{XV_{\mathrm{rel}}}$ be the joint state of latent and regulator-released prophecy.
Define
\[
    L(\rho) = I(X ; V_{\mathrm{rel}}).
\]
\end{definition}

Ethical feasibility requires $L(\rho)$ to be bounded.

\begin{definition}[Leakage Constraint]
A regulator obeys the leakage constraint for parameter $\epsilon > 0$ if
\[
    L(\rho) \le \epsilon
\]
for all states encountered under repeated application of $\Phi^{\mathcal{Q}}$.
\end{definition}

\subsection{Regulator Optimization Problem}

The regulator controls $\lambda$ (noise level) and $\alpha$ (measurement sharpness).

\begin{definition}[Regulatory Action Space]
The regulator’s admissible controls are
\[
    (\lambda, \alpha) \in \mathcal{U}
    =
    [0,1] \times [0,1].
\]
\end{definition}

To minimize long-term harm, we introduce a value function.

\begin{definition}[Ethical Value Function]
For discount factor $\gamma \in (0,1)$,
\[
    V(\rho)
    =
    \min_{(\lambda,\alpha)\in \mathcal{U}}
    \left[
        H(\rho)
        +
        \gamma\,
        \mathbb{E}\Big[ V\big(\Phi^{\mathcal{Q}}_{\lambda,\alpha}(\rho)\big) \Big]
    \right].
\]
\end{definition}

Here $\Phi^{\mathcal{Q}}_{\lambda,\alpha}$ denotes the prophecy--response operator with explicit
dependence on regulator-controlled parameters.

\subsection{Bellman Equation and Existence}

\begin{theorem}[Bellman Equation]
The value function satisfies
\[
    V(\rho)
    =
    \min_{(\lambda,\alpha) \in \mathcal{U}}
    \left[
        H(\rho)
        +
        \gamma
        V(\Phi^{\mathcal{Q}}_{\lambda,\alpha}(\rho))
    \right].
\]
\end{theorem}

\begin{proof}[Sketch]
This is a standard dynamic programming principle extended to operator spaces. The finite
dimensionality of $\mathcal{H}$ ensures compactness of $\mathcal{D}(\mathcal{H})$, and continuity of
$\Phi^{\mathcal{Q}}_{\lambda,\alpha}$ ensures existence of minimizers on the compact action set
$\mathcal{U}$.
\end{proof}

\begin{theorem}[Existence of Optimal Regulator]
There exists at least one measurable selector $(\lambda^\ast,\alpha^\ast)$ achieving the minimum in the
Bellman equation for all $\rho$.
\end{theorem}

\begin{proof}[Sketch]
Compactness of $\mathcal{U}$ and continuity of $H$ and $\Phi^{\mathcal{Q}}_{\lambda,\alpha}$ ensure the
existence of minimizers by the Weierstrass theorem.
\end{proof}

This establishes the formal ethical layer governing the quantum prophetic world.

% -------------------------------------------------------
% SECTION 6: COMPUTABILITY-THEORETIC LIMITS OF PROPHECY
% -------------------------------------------------------
\section{Computability-Theoretic Limits of Prophecy}

Quantum prophetic systems inherit the classical difficulty that certain future events may encode
computationally undecidable behavior. In this section we formalize minimal conditions under which
prophecy---even approximate prophecy---becomes impossible due to computability limits. These
results form the theoretical boundary of any reflexive prophetic system.

\subsection{Prophetic Decision Problems}

Let $\mathcal{P}$ denote the set of all possible sequences obtained through repeated application of
$\Phi^{\mathcal{Q}}$:
\[
    \rho_{t+1} = \Phi^{\mathcal{Q}}(\rho_t), \qquad t \in \mathbb{N}.
\]

\begin{definition}[Crash-Event Predicate]
Define the crash predicate
\[
    C(\rho) =
    \begin{cases}
        1, & \text{if } P_{\mathrm{crash}}(\rho) > \tfrac{1}{2}, \\
        0, & \text{otherwise}.
    \end{cases}
\]
\end{definition}

\begin{definition}[Prophecy Decision Problem]
Given $(\rho_0,\Phi^{\mathcal{Q}})$, determine whether there exists $t$ such that $C(\rho_t)=1$.
\end{definition}

We now investigate the computability-theoretic complexity of this decision problem.

\subsection{Encoding Computational Processes}

Quantum channels can simulate classical computation via reversible embeddings. Let $U_M$ denote
a unitary that simulates a classical Turing machine $M$ in its computational basis.

\begin{lemma}[Computational Embedding]
For any classical Turing machine $M$, there exists a CPTP map $\mathcal{E}_M$ acting on an extended
Hilbert space such that the evolution of $\rho_t$ under $\Phi^{\mathcal{Q}}$ encodes the computation
history of $M$.
\end{lemma}

\begin{proof}[Sketch]
The Stinespring dilation theorem guarantees that any classical computation can be embedded
in a unitary evolution on a larger Hilbert space. The prophecy--response operator can thus be
augmented with auxiliary registers implementing reversible computational steps.
\end{proof}

\subsection{Undecidability of Perfect Prophecy}

\begin{theorem}[Undecidability of Perfect Prophecy]
There is no algorithm that, given $(\rho_0,\Phi^{\mathcal{Q}})$, decides whether a crash event occurs in
finite time for all such instances.
\end{theorem}

\begin{proof}[Sketch]
By the computational embedding lemma, we can construct $\mathcal{E}_M$ such that the latent
state registers encode whether $M$ halts. Then $C(\rho_t)=1$ if and only if $M$ halts. Deciding
crash occurrence would thus solve the Halting Problem, which is impossible.
\end{proof}

\subsection{Approximate Prophecy and Limitations}

Exact prophecy is impossible in general. However, approximate prophecy remains feasible if
information is restricted to computably decidable aspects of the underlying state evolution.

\begin{definition}[Approximate Prophecy]
Let $\delta > 0$. An approximate prophecy is a sequence $\{\tilde{\rho}_t\}$ such that
\[
    \|\tilde{\rho}_t - \rho_t\|_1 \le \delta
\]
for all $t$.
\end{definition}

\begin{theorem}[Limit of Approximate Prophecy]
Even approximate prophecy is undecidable in general if $\delta$ is required to distinguish crash vs.
non-crash states in cases reducible to halting behavior.
\end{theorem}

\begin{proof}[Sketch]
If distinguishing between crash and non-crash requires distinguishing between halting and
non-halting computations encoded in latent registers, then any $\delta$-approximation that reliably
determines crash outcomes solves the Halting Problem.
\end{proof}

\subsection{Decidability Frontiers}

Despite general impossibility, decidability is restored under certain constraints.

\begin{theorem}[Decidability Frontier]
The prophecy decision problem becomes decidable when
\[
    \norm{\mathcal{E}(\rho_1) - \mathcal{E}(\rho_2)}_1 \le k \norm{\rho_1 - \rho_2}_1
\]
with $k < 1$ uniformly and when the action space of the agent and regulator is finite.
\end{theorem}

\begin{proof}[Sketch]
Under strict contraction, the sequence $\rho_t$ converges to a unique fixed point $\rho^\ast$, and
the crash predicate reduces to the evaluation of $C(\rho^\ast)$, which is decidable by direct
inspection.
\end{proof}

This section establishes the theoretical limits of prophecy. While perfect and universal prophecy
is impossible, practical quantum prophecy remains feasible when confined to contractive, ethically
constrained regimes.
