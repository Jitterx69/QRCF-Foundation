% -------------------------------------------------------
% SECTION 7: MAIN THEOREM SERIES
% -------------------------------------------------------

In this section we develop and consolidate the fundamental theoretical results governing the
quantum prophecy--response operator $\Phi^{\mathcal{Q}}$. These theorems address the existence of
fixed points, uniqueness under contraction, ethical feasibility conditions, stability boundaries, and
the limits imposed by computational complexity. Together they form the mathematical backbone
of quantum reflexive control.

Throughout, $\mathcal{H} = \mathcal{H}_X \otimes \mathcal{H}_V$ denotes a fixed finite-dimensional
Hilbert space and $\mathcal{D}(\mathcal{H})$ the set of density operators on $\mathcal{H}$.

\subsection{Existence of Prophetic Equilibria}

We begin by establishing fundamental conditions for the existence of a reflexive equilibrium, i.e.,
a state $\rho^\ast$ satisfying $\Phi^{\mathcal{Q}}(\rho^\ast) = \rho^\ast$.

\begin{theorem}[Existence of Fixed Points]
\label{thm:existence-fixed-point}
The operator $\Phi^{\mathcal{Q}} : \mathcal{D}(\mathcal{H}) \to \mathcal{D}(\mathcal{H})$ admits at least
one fixed point.
\end{theorem}

\begin{proof}
The set $\mathcal{D}(\mathcal{H})$ is a compact convex subset of a finite-dimensional vector
space. The map $\Phi^{\mathcal{Q}}$ is continuous because it is a composition of continuous maps
($M$, $\rho_{\mathrm{rel}}$, $\pi$, $\mathcal{E}$). By the Brouwer fixed-point theorem, any continuous
self-map on a compact convex set admits at least one fixed point.
\end{proof}

This result holds independently of contraction or ethical feasibility constraints.

\subsection{Uniqueness via Contraction}

We strengthen existence to uniqueness when the operator exhibits contractive behavior.

\begin{theorem}[Uniqueness Under Contraction]
\label{thm:unique-fixed-point}
Assume there exists $k < 1$ such that
\[
    \|\Phi^{\mathcal{Q}}(\rho_1) - \Phi^{\mathcal{Q}}(\rho_2)\|_1
    \le
    k \|\rho_1 - \rho_2\|_1
\]
for all $\rho_1,\rho_2 \in \mathcal{D}(\mathcal{H})$. Then $\Phi^{\mathcal{Q}}$ has a unique fixed point
$\rho^\ast$, and for any initial state $\rho_0$,
\[
    \lim_{t\to\infty} (\Phi^{\mathcal{Q}})^t(\rho_0) = \rho^\ast.
\]
\end{theorem}

\begin{proof}
This follows directly from the Banach fixed-point theorem. Contractivity ensures that the
sequence $\rho_{t+1} = \Phi^{\mathcal{Q}}(\rho_t)$ converges to a unique point.
\end{proof}

\begin{remark}
The contraction constant $k$ can be tuned through regulator parameters $(\lambda,\alpha)$.
Noise (increasing $\lambda$) strengthens contraction, while measurement sharpness (increasing
$\alpha$) weakens it.
\end{remark}

\subsection{Stability of Prophetic Equilibria}

The stability of equilibria under small perturbations is essential for reliably deploying quantum
prophetic systems.

\begin{definition}[Stability]
A fixed point $\rho^\ast$ of $\Phi^{\mathcal{Q}}$ is \emph{stable} if for every $\epsilon > 0$ there exists
$\delta > 0$ such that
\[
    \|\rho - \rho^\ast\|_1 < \delta
    \quad\Rightarrow\quad
    \|\Phi^{\mathcal{Q}}(\rho) - \rho^\ast\|_1 < \epsilon.
\]
\end{definition}

\begin{theorem}[Stability Under Contraction]
Any fixed point arising under the contraction conditions of
Theorem~\ref{thm:unique-fixed-point} is stable.
\end{theorem}

\begin{proof}
Immediate from contraction: the operator reduces distances monotonically.
\end{proof}

\subsection{Ethical Feasibility and Harm Minimization}

We now provide theoretical guarantees for the ethical control framework introduced in
Section~5.

\begin{theorem}[Feasible Ethical Control]
\label{thm:ethical-feasibility}
Let $\epsilon > 0$ be a leakage threshold. Suppose $\mathcal{U}$ contains parameters
$(\lambda,\alpha)$ such that:
\begin{itemize}[leftmargin=1cm]
    \item $L(\rho) \le \epsilon$ for all reachable states $\rho$,
    \item $\Phi^{\mathcal{Q}}_{\lambda,\alpha}$ is contractive with constant $k < 1$.
\end{itemize}
Then there exists an ethically feasible regulator achieving minimal harm.
\end{theorem}

\begin{proof}[Sketch]
The leakage constraint restricts the action space. Compactness of the set of feasible
parameters ensures existence of minimizers. Contraction ensures that all trajectories converge to
a unique equilibrium, allowing evaluation of the Bellman objective via fixed points.
\end{proof}

\begin{theorem}[Ethical Fixed Point]
Under the assumptions of Theorem~\ref{thm:ethical-feasibility}, there exists a unique fixed point
$\rho^\ast$ satisfying both:
\[
    \rho^\ast = \Phi^{\mathcal{Q}}_{\lambda^\ast,\alpha^\ast}(\rho^\ast),
\qquad
    L(\rho^\ast) \le \epsilon.
\]
\end{theorem}

\begin{proof}
Existence follows from Brouwer; uniqueness follows from contraction.
\end{proof}

\subsection{Prophetic Impossibility Under Undecidability}

We now formalize the conditions under which prophecy is impossible due to computational
constraints.

\begin{theorem}[Undecidability of Crash Prediction]
Let $\Phi^{\mathcal{Q}}$ encode an arbitrary Turing computation in its latent register. Then there is no
algorithm that decides whether
\[
    P_{\mathrm{crash}}\big((\Phi^{\mathcal{Q}})^t(\rho_0)\big) > \tfrac{1}{2}
\]
for some $t \in \mathbb{N}$.
\end{theorem}

\begin{proof}
The existence of such an algorithm would solve the Halting Problem by reduction. See
Section~6 for details.
\end{proof}

\subsection{Ethical Boundaries from Computability}

Ethical control interacts nontrivially with undecidability.

\begin{definition}[Ethically Impossible Regime]
A regulatory regime is ethically impossible if minimizing $H(\rho)$ requires distinguishing
between states encoding halting vs.\ non-halting computations.
\end{definition}

\begin{theorem}[Ethical Impossibility Theorem]
If achieving $H(\rho) < H_0$ for some threshold $H_0$ requires resolving crash outcomes encoded
by non-halting computations, then no regulator can achieve $H(\rho) < H_0$.
\end{theorem}

\begin{proof}[Sketch]
If $H(\rho)$ can only be minimized by obtaining prophetic information that distinguishes halting
from non-halting behavior, then any such regulator must solve the Halting Problem. This is
impossible.
\end{proof}

\subsection{Stability Boundaries}

Finally, we examine the interface between computation and dynamical stability.

\begin{theorem}[Stability Frontier]
If $\Phi^{\mathcal{Q}}$ simulates a universal Turing machine on a subsystem, stability of the fixed
point cannot be guaranteed in general.
\end{theorem}

\begin{proof}[Sketch]
Arbitrarily small changes to initial conditions can cause divergence of computational paths,
preventing contraction. Thus, the operator cannot be globally stable in trace norm.
\end{proof}

\begin{corollary}[Decidability and Stability Coincide in Contractive Regimes]
If $\Phi^{\mathcal{Q}}$ is contractive, then prophecy is decidable (Section~6) and the equilibrium is
stable (this section).
\end{corollary}

This completes the theoretical framework for existence, stability, ethical feasibility, and
computational boundaries of quantum prophecy.
