% -------------------------------------------------------
% SECTION 1: INTRODUCTION
% -------------------------------------------------------

Predictive systems in classical settings are traditionally conceived as passive observers of the world:
they estimate or forecast future states without altering the underlying dynamics. However, in
many real and theoretical environments, the act of revealing a prediction itself affects the world it
describes. This phenomenon---which we refer to as \emph{prophecy} rather than prediction---is
endogenously reflexive and can fundamentally reshape causal, informational, and strategic structures.

Classical models of prophecy rely on stochastic or game-theoretic frameworks, whereas quantum
prophecy requires a more sophisticated formulation: information is encoded in density operators,
its extraction induces unavoidable disturbance, and attempts to disseminate prophetic information
are constrained by quantum no-cloning principles and entropic limits. When agents respond to
released prophecy and environments evolve through completely positive trace-preserving (CPTP)
maps, the prediction process becomes inseparable from the world it influences.

This monograph introduces a formal mathematical theory of \emph{Quantum Reflexive Control}
(QRC). We construct a composite operator
\[
    \Phi^{\mathcal{Q}} : \mathcal{D}(\mathcal{H}) \to \mathcal{D}(\mathcal{H}),
\]
acting on the space of density operators of a finite-dimensional Hilbert space~$\mathcal{H}$,
representing the combined evolution of prophetic measurement, regulator-controlled information
release, agent responses, and environmental CPTP dynamics. The operator $\Phi^{\mathcal{Q}}$ is a
quantum analogue of classical reflexive prophecy-response maps, enriched with measurement-induced
disturbance and constrained by ethical information bottlenecks.

We provide a complete formal construction:
\begin{enumerate}[label=(\roman*)]
    \item a definition of the quantum prophetic world and its memory structure;
    \item an explicit parameterization of measurement instruments and regulator channels;
    \item a definition of agent response maps represented as conditional unitary channels;
    \item a full operator-theoretic description of $\Phi^{\mathcal{Q}}$;
    \item proofs that $\Phi^{\mathcal{Q}}$ is CPTP under mild assumptions;
    \item sufficient conditions for the existence of reflexive fixed points;
    \item ethical feasibility conditions via entropic leakage bounds and harm functionals;
    \item operator-theoretic constraints arising from computational impossibility.
\end{enumerate}

The remainder of this work is structured as follows.
Section~2 establishes mathematical preliminaries in Hilbert spaces, density operators, CPTP maps,
and measurement theory. Section~3 introduces the quantum prophetic world and its state structure.
Section~4 defines the prophecy--response operator and derives its fundamental properties. Section~5
develops the ethical control layer and corresponding optimization problem. Section~6 introduces
computability-theoretic boundaries for prophecy. Section~7 presents a unified theorem series on
existence, impossibility, stability, and ethical constraints. Section~8 provides an explicit example with
fully worked CPTP maps, POVMs, and regulator channels. The appendices contain additional proofs,
operator norm bounds, and entropy inequalities.

This monograph lays the formal foundation for the design of Quantum Reflexive Control Engines
(QRCEs), systems capable of stabilizing reflexive quantum prophecy under physical, ethical, and
computational constraints.

% -------------------------------------------------------
% SECTION 2: MATHEMATICAL PRELIMINARIES
% -------------------------------------------------------
\section{Mathematical Preliminaries}

This section introduces the operator-theoretic and quantum-mechanical definitions required for
the construction of $\Phi^{\mathcal{Q}}$. Throughout, we assume all Hilbert spaces are finite-dimensional.

\subsection{Hilbert Spaces and Operators}

Let $\mathcal{H}$ be a finite-dimensional Hilbert space equipped with inner product
$\langle \cdot, \cdot \rangle$. We denote by $\mathcal{B}(\mathcal{H})$ the space of linear operators on
$\mathcal{H}$, and by $\mathcal{D}(\mathcal{H})$ the set of density operators:
\[
    \mathcal{D}(\mathcal{H}) = 
    \left\{
        \rho \in \mathcal{B}(\mathcal{H}) :
        \rho \succeq 0,\;
        \mathrm{Tr}(\rho) = 1
    \right\}.
\]

\begin{definition}[Partial Trace]
Let $\mathcal{H} = \mathcal{H}_A \otimes \mathcal{H}_B$. The partial trace over subsystem~$B$ is the
unique linear map
\[
    \mathrm{Tr}_B : \mathcal{B}(\mathcal{H}) \to \mathcal{B}(\mathcal{H}_A)
\]
satisfying
\[
    \mathrm{Tr}_B(A \otimes B) = (\mathrm{Tr}\,B)\, A
\]
for all $A \in \mathcal{B}(\mathcal{H}_A)$ and $B \in \mathcal{B}(\mathcal{H}_B)$.
\end{definition}

\begin{definition}[Von Neumann Entropy]
For $\rho \in \mathcal{D}(\mathcal{H})$, the von Neumann entropy is
\[
    S(\rho) = - \mathrm{Tr}(\rho \log \rho).
\]
\end{definition}

\subsection{Positive and Completely Positive Maps}

A linear operator $T : \mathcal{B}(\mathcal{H}) \to \mathcal{B}(\mathcal{H})$ is \emph{positive} if
$T(X) \succeq 0$ whenever $X \succeq 0$.

\begin{definition}[Complete Positivity]
A linear map $T : \mathcal{B}(\mathcal{H}) \to \mathcal{B}(\mathcal{K})$ is \emph{completely positive}
(CP) if for all $n \in \mathbb{N}$, the map
\[
    T \otimes \mathrm{id}_{n} : 
    \mathcal{B}(\mathcal{H} \otimes \mathbb{C}^n)
    \to 
    \mathcal{B}(\mathcal{K} \otimes \mathbb{C}^n)
\]
is positive.
\end{definition}

\begin{definition}[CPTP Map]
A completely positive map $T$ is \emph{trace-preserving} (TP) if
$\mathrm{Tr}(T(X)) = \mathrm{Tr}(X)$ for all $X$. A \emph{CPTP map} is a completely positive,
trace-preserving map.
\end{definition}

\begin{proposition}[Kraus Representation]
A map $T$ is CPTP if and only if there exist operators $\{K_i\}$ such that
\[
    T(\rho) = \sum_i K_i \rho K_i^\dagger,
    \qquad
    \sum_i K_i^\dagger K_i = \id.
\]
\end{proposition}

\subsection{Quantum Measurements}

\begin{definition}[POVM]
A positive operator-valued measure (POVM) on $\mathcal{H}$ is a finite collection of operators
$\{E_i\} \subset \mathcal{B}(\mathcal{H})$ such that
\[
    E_i \succeq 0,\qquad \sum_i E_i = \id.
\]
\end{definition}

A POVM defines measurement statistics but not state update. For that, we require instruments.

\begin{definition}[Quantum Instrument]
A quantum instrument is a set of CPTP maps $\{\mathcal{M}_i\}$ such that
\[
    \sum_i \mathcal{M}_i
\]
is CPTP. The post-measurement state given outcome $i$ is
\[
    \rho_i = \frac{\mathcal{M}_i(\rho)}{\mathrm{Tr}(\mathcal{M}_i(\rho))}.
\]
\end{definition}

In this work, we use Kraus operators $\{M_i\}$ satisfying $E_i = M_i^\dagger M_i$.

\subsection{Trace Norm and Contractions}

\begin{definition}[Trace Norm]
For $X \in \mathcal{B}(\mathcal{H})$, the trace norm is
\[
    \|X\|_1 = \mathrm{Tr}\sqrt{X^\dagger X}.
\]
\end{definition}

\begin{definition}[Contraction]
A map $T : \mathcal{D}(\mathcal{H}) \to \mathcal{D}(\mathcal{H})$ is a contraction if there exists 
$0 \leq k < 1$ such that
\[
    \|T(\rho_1) - T(\rho_2)\|_1 \le k\, \|\rho_1 - \rho_2\|_1
\]
for all density operators $\rho_1, \rho_2$.
\end{definition}

Contractions guarantee unique fixed points by the Banach fixed-point theorem.

This completes the preliminary operator-theoretic foundation required for the construction of
the prophecy--response operator in Section~4.
