% -------------------------------------------------------
% APPENDIX A: EXPANDED PROOFS
% -------------------------------------------------------

This appendix contains detailed proofs and expansions of the statements presented in the main
text. All Hilbert spaces are finite-dimensional unless otherwise noted.

% -------------------------------------------------------
\subsection{Proof that the Measurement Instrument $M$ is CPTP}

Recall that
\[
    M(\rho)
    =
    \sum_{i=0}^1
    (\id_X \otimes M_i)\, \rho\, (\id_X \otimes M_i^\dagger),
\]
where $M_i = \sqrt{E_i}$ and $E_0 + E_1 = \id_V$.

\begin{proof}
Since $E_i \succeq 0$, each $M_i$ is well-defined and satisfies $M_i^\dagger M_i = E_i$.
The map $M(\cdot)$ is of the standard Kraus form
\[
    M(\cdot) = \sum_j A_j(\cdot)A_j^\dagger,
\]
where $A_i = \id_X \otimes M_i$. Thus $M$ is CP.

Trace preservation follows because
\[
\sum_{i=0}^1 A_i^\dagger A_i
= \sum_{i=0}^1 (\id_X\otimes M_i^\dagger)(\id_X\otimes M_i)
= \id_X\otimes \sum_{i=0}^1 M_i^\dagger M_i
= \id_X\otimes \id_V
= \id.
\]
Hence $M$ is CPTP.
\end{proof}

% -------------------------------------------------------
\subsection{Proof that the Regulator Channel $\rho_{\mathrm{rel}}$ is CPTP}

The regulator applies the depolarizing channel on $\mathcal{H}_V$:
\[
\rho_{\mathrm{rel}}(\sigma)
= (1-\lambda)\sigma + \lambda \frac{\id_V}{2}\otimes \Tr_V(\sigma).
\]

\begin{proof}
Define the CPTP maps
\[
\mathcal{D}_0(\sigma) = \sigma,\qquad
\mathcal{D}_1(\sigma) = \frac{\id_V}{2}\otimes \Tr_V(\sigma).
\]
The regulator channel is the convex combination
\[
\rho_{\mathrm{rel}} = (1-\lambda)\mathcal{D}_0 + \lambda\mathcal{D}_1.
\]
Convex combinations of CPTP maps are CPTP. Therefore $\rho_{\mathrm{rel}}$ is CPTP.
\end{proof}

% -------------------------------------------------------
\subsection{Proof that the Agent Response Map $\pi$ is CPTP}

The agent response map is defined by
\[
\pi(\rho)
=
\sum_{i=0}^1
(U_i\otimes\id_V)\,
(\id_X\otimes M_i)\,\rho\,(\id_X\otimes M_i^\dagger)
\,(U_i^\dagger\otimes\id_V).
\]

\begin{proof}
Each term in the sum has the form
\[
    \rho \mapsto (U_i\otimes\id_V)
                 (\id_X\otimes M_i)\,\rho\,
                 (\id_X\otimes M_i^\dagger)
                 (U_i^\dagger\otimes\id_V).
\]
This is CP because it is a composition of CP maps:
\begin{itemize}
    \item Left multiplication by $(\id\otimes M_i)$ is CP;
    \item Right multiplication by $(\id\otimes M_i^\dagger)$ is CP;
    \item Conjugation by a unitary $(U_i\otimes\id)$ is CP.
\end{itemize}
Trace preservation holds because
\[
\sum_{i=0}^1 (U_i\otimes\id)(\id\otimes M_i^\dagger M_i)(U_i^\dagger\otimes\id)
=
\sum_{i=0}^1 (\id_X\otimes E_i)
= \id,
\]
as $E_0+E_1=\id_V$.
\end{proof}

% -------------------------------------------------------
\subsection{Expanded Proof that $\Phi^{\mathcal{Q}}$ is CPTP}

\[
    \Phi^{\mathcal{Q}}
    =
    \mathcal{E}
    \circ
    \pi
    \circ
    \rho_{\mathrm{rel}}
    \circ
    M.
\]

\begin{proof}
From A.1, A.2, and A.3 we know that each map in the composition is CPTP. CPTP maps are closed
under composition. Therefore $\Phi^{\mathcal{Q}}$ is CPTP.
\end{proof}

% -------------------------------------------------------
\subsection{Expanded Proof of Fixed-Point Existence}

\begin{theorem*}[Theorem 1 (Existence of Fixed Points)]
$\Phi^{\mathcal{Q}}$ admits at least one fixed point in $\mathcal{D}(\mathcal{H})$.
\end{theorem*}

\begin{proof}
The space $\mathcal{D}(\mathcal{H})$ is compact and convex in the Euclidean topology. Since
$\Phi^{\mathcal{Q}}$ is continuous (all matrix operations are continuous), it follows from the Brouwer
fixed-point theorem that $\Phi^{\mathcal{Q}}$ has at least one fixed point.
\end{proof}

% -------------------------------------------------------
\subsection{Expanded Proof of Contraction Uniqueness}

\begin{theorem*}[Theorem 2 (Uniqueness Under Contraction)]
If
\[
\|\Phi^{\mathcal{Q}}(\rho_1)-\Phi^{\mathcal{Q}}(\rho_2)\|_1
\le k\|\rho_1 - \rho_2\|_1,
\quad
k<1,
\]
then there exists a unique fixed point $\rho^\ast$, and
\(
\displaystyle \lim_{t\to\infty} (\Phi^{\mathcal{Q}})^t(\rho_0)=\rho^\ast.
\)
\end{theorem*}

\begin{proof}
By assumption $\Phi^{\mathcal{Q}}$ is a contraction on the complete metric space
$(\mathcal{D}(\mathcal{H}),\|\cdot\|_1)$. The Banach fixed-point theorem guarantees existence and
uniqueness of the fixed point and geometric convergence of iterates toward it.
\end{proof}

% -------------------------------------------------------
\subsection{Expanded Proof of Ethical Feasibility Theorems}

\begin{theorem*}[Ethical Feasibility (Theorem 3)]
If $L(\rho)\le\epsilon$ for all reachable states and $\Phi^{\mathcal{Q}}_{\lambda,\alpha}$ is contractive,
then an optimal regulator exists.
\end{theorem*}

\begin{proof}
Let $\mathcal{U}_\epsilon \subseteq [0,1]^2$ denote the feasible parameter set satisfying the leakage
constraint. Since $L$ is continuous and $[0,1]^2$ is compact, the feasible set is closed and compact.
The Bellman objective
\[
H(\rho) + \gamma\,V(\Phi^{\mathcal{Q}}_{\lambda,\alpha}(\rho))
\]
is continuous in $(\lambda,\alpha)$ and attains a minimum on a compact set. Hence an optimal
regulator strategy exists.
\end{proof}

\begin{theorem*}[Ethical Fixed Point (Theorem 4)]
Under the above assumptions, the system admits a unique fixed point $\rho^\ast$ satisfying
\[
    \rho^\ast = \Phi^{\mathcal{Q}}_{\lambda^\ast,\alpha^\ast}(\rho^\ast),\qquad L(\rho^\ast)\le \epsilon.
\]
\end{theorem*}

\begin{proof}
Existence follows from Brouwer because $\Phi^{\mathcal{Q}}_{\lambda^\ast,\alpha^\ast}$ is continuous.
Uniqueness follows from the contraction assumption.
\end{proof}

% -------------------------------------------------------
\subsection{Formalization of the Computability Embedding Lemma}

\begin{lemma*}[Computational Embedding]
For any classical Turing machine $M$, there exists a CPTP map $\mathcal{E}_M$ such that the
iteration $\rho_{t+1}=\Phi^{\mathcal{Q}}(\rho_t)$ encodes the computation history of $M$ in a
tensor-factor subsystem of $\mathcal{H}$.
\end{lemma*}

\begin{proof}
Any reversible classical computation can be embedded into a unitary operation $U_M$ on an
extended Hilbert space. For each computational step, extend $\Phi^{\mathcal{Q}}$ by tensoring with
ancilla registers encoding tape, head position, and internal states. The composite channel
\[
\mathcal{E}_M(\rho) = U_M (\rho\otimes\ket{0}\!\bra{0}) U_M^\dagger
\]
is CPTP. Since $\Phi^{\mathcal{Q}}$ is CPTP and extensible, the iteration
$\rho_{t+1}=\Phi^{\mathcal{Q}}(\rho_t)$ reproduces $M$'s computational trajectory.
\end{proof}

% -------------------------------------------------------
\subsection{Expanded Proof of the Prophetic Impossibility Theorem}

\begin{theorem*}[Undecidability of Perfect Prophecy]
Determining whether $P_{\mathrm{crash}}((\Phi^{\mathcal{Q}})^t(\rho_0))>\tfrac{1}{2}$ for some
$t\in\mathbb{N}$ is undecidable in general.
\end{theorem*}

\begin{proof}
Given any Turing machine $M$, construct $\mathcal{E}_M$ as above so that $M$ halts if and only
if the latent register enters a designated crash state. Then:
\[
M\ \text{halts} \iff 
\exists t:\ P_{\mathrm{crash}}\big((\Phi^{\mathcal{Q}})^t(\rho_0)\big)>\tfrac{1}{2}.
\]
If such crash detection were decidable, the Halting Problem would be decidable. Contradiction.
\end{proof}
