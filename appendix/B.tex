% -------------------------------------------------------
% APPENDIX B: OPERATOR NORM BOUNDS
% -------------------------------------------------------

This appendix establishes operator-norm inequalities used throughout the monograph. These
results justify contraction bounds, measurement-disturbance inequalities, and continuity arguments
for the prophecy--response operator $\Phi^{\mathcal{Q}}$.

Throughout, $\|\cdot\|_1$ denotes the trace norm, $\|\cdot\|_2$ the Hilbert--Schmidt norm,
$\|\cdot\|_\infty$ the operator norm, and $\mathcal{D}(\mathcal{H})$ the space of density operators on
a finite-dimensional Hilbert space $\mathcal{H}$.

% -------------------------------------------------------
\subsection{Basic Relationships Between Operator Norms}

\begin{lemma}[Norm Dominance]
For all $A\in\mathcal{B}(\mathcal{H})$,
\[
    \|A\|_1 \le \sqrt{d}\,\|A\|_2 \le d\,\|A\|_\infty,
\]
where $d=\dim(\mathcal{H})$.
\end{lemma}

\begin{proof}
The inequalities $\|A\|_1 \le \sqrt{d}\|A\|_2$ and $\|A\|_2 \le \sqrt{d}\|A\|_\infty$ follow from
standard Schatten-norm relationships.
\end{proof}

\begin{corollary}
For density operators $\rho,\sigma$,
\[
    \|\rho - \sigma\|_1 \le 2.
\]
\end{corollary}

This bound will be used for global Lipschitz estimates.

% -------------------------------------------------------
\subsection{Bounds for CPTP Maps}

\begin{lemma}[Non-expansiveness of Quantum Channels]
\label{lem:nonexp}
For any CPTP map $\mathcal{E}$ and any density matrices $\rho,\sigma$,
\[
    \|\mathcal{E}(\rho) - \mathcal{E}(\sigma)\|_1 \le \|\rho - \sigma\|_1.
\]
\end{lemma}

\begin{proof}
This is a standard result in quantum information theory: CPTP maps are contractions in trace
distance because the trace norm is monotone under quantum operations.
\end{proof}

\begin{remark}
Lemma~\ref{lem:nonexp} applies to $M$, $\rho_{\mathrm{rel}}$, $\pi$, and $\mathcal{E}$ individually.
\end{remark}

% -------------------------------------------------------
\subsection{Measurement-Disturbance Inequalities}

We derive bounds relating measurement sharpness $\alpha$ to disturbance.

Recall the POVM elements
\[
E_0 = (1-\alpha)\tfrac{\id}{2} + \alpha\ket{0}\!\bra{0},\qquad
E_1 = (1-\alpha)\tfrac{\id}{2} + \alpha\ket{1}\!\bra{1},
\]
with associated Kraus operators $M_i=\sqrt{E_i}$.

\begin{lemma}[Disturbance Bound]
For any $\rho\in\mathcal{D}(\mathcal{H})$,
\[
    \| \rho - M(\rho) \|_1 \le 2\alpha.
\]
\end{lemma}

\begin{proof}
Since $M$ interpolates between the identity ($\alpha=0$) and a projective measurement
($\alpha=1$), its deviation from the identity map is at most $\alpha$ in each measurement branch.
Estimating the sum yields a bound of $2\alpha$.
\end{proof}

\begin{corollary}
If $\alpha$ is small, the measurement induces minimal disturbance:
\[
\alpha \ll 1 \quad\Rightarrow\quad \|\rho - M(\rho)\|_1 = O(\alpha).
\]
\end{corollary}

This bound supports ethical constraints on disturbance penalties.

% -------------------------------------------------------
\subsection{Regulator Bounds for the Depolarizing Channel}

The regulator applies
\[
\rho_{\mathrm{rel}}(\sigma) = (1-\lambda)\sigma + \lambda\tfrac{\id}{2}\otimes\Tr_V(\sigma).
\]

\begin{lemma}[Depolarizing Contraction Bound]
For all $\sigma,\tau\in\mathcal{D}(\mathcal{H})$,
\[
\|\rho_{\mathrm{rel}}(\sigma) - \rho_{\mathrm{rel}}(\tau)\|_1
\le (1-\lambda)\|\sigma - \tau\|_1.
\]
\end{lemma}

\begin{proof}
The second component of the depolarizing channel is independent of the input, so differences
cancel:
\[
\rho_{\mathrm{rel}}(\sigma) - \rho_{\mathrm{rel}}(\tau)
= (1-\lambda)(\sigma - \tau).
\]
Taking trace norm gives the result.
\end{proof}

Thus $\lambda$ directly controls contraction strength.

% -------------------------------------------------------
\subsection{Bounds for the Agent Response Operator $\pi$}

\begin{lemma}[Unitary Preservation of Trace Norm]
For any unitary $U$ and operator $A$,
\[
\|UAU^\dagger\|_1 = \|A\|_1.
\]
\end{lemma}

Since $\pi$ is built from conjugation by unitaries and the measurement instrument $M$:

\begin{proposition}
\[
\|\pi(\rho) - \pi(\sigma)\|_1 \le \|\rho - \sigma\|_1.
\]
\end{proposition}

\begin{proof}
Apply Lemma~\ref{lem:nonexp} to the measurement instrument followed by trace-norm preservation
of unitary conjugation.
\end{proof}

% -------------------------------------------------------
\subsection{Global Lipschitz Bound for $\Phi^{\mathcal{Q}}$}

We derive an upper bound on the Lipschitz constant of the entire operator.

\begin{theorem}[Lipschitz Constant of the Prophecy--Response Operator]
Let
\[
\Phi^{\mathcal{Q}}
= \mathcal{E}\circ\pi\circ\rho_{\mathrm{rel}}\circ M.
\]
Then for all $\rho,\sigma$,
\[
\|\Phi^{\mathcal{Q}}(\rho) - \Phi^{\mathcal{Q}}(\sigma)\|_1
\le (1-\lambda)\|\rho - \sigma\|_1.
\]
\end{theorem}

\begin{proof}
Apply the following inequalities sequentially:
\[
\|M(\rho) - M(\sigma)\|_1 \le \|\rho - \sigma\|_1,
\]
\[
\|\rho_{\mathrm{rel}}(A) - \rho_{\mathrm{rel}}(B)\|_1 \le (1-\lambda)\|A - B\|_1,
\]
\[
\|\pi(A) - \pi(B)\|_1 \le \|A - B\|_1,
\]
\[
\|\mathcal{E}(A) - \mathcal{E}(B)\|_1 \le \|A - B\|_1.
\]
Multiplying the inequalities gives the result.
\end{proof}

\begin{corollary}[Sufficient Condition for Contraction]
If $\lambda>0$, then $\Phi^{\mathcal{Q}}$ is strictly contractive:
\[
\|\Phi^{\mathcal{Q}}(\rho)-\Phi^{\mathcal{Q}}(\sigma)\|_1 \le (1-\lambda)\|\rho - \sigma\|_1.
\]
\end{corollary}

This is the key condition for unique reflexive equilibria.

% -------------------------------------------------------
\subsection{Bound on the Difference Between Fixed Points Under Parameter Perturbation}

Let $\Phi^{\mathcal{Q}}_{\lambda,\alpha}$ denote the operator with explicit parameter dependence.

\begin{theorem}[Fixed-Point Sensitivity]
Suppose $\Phi^{\mathcal{Q}}_{\lambda,\alpha}$ and $\Phi^{\mathcal{Q}}_{\lambda',\alpha'}$ are contractive
with constants $k,k'<1$. Let $\rho^\ast$ and ${\rho^\ast}'$ be their fixed points. Then
\[
\|\rho^\ast - {\rho^\ast}'\|_1
\le \frac{1}{1-k}\,
\|\Phi^{\mathcal{Q}}_{\lambda,\alpha}(\rho^\ast)
 - \Phi^{\mathcal{Q}}_{\lambda',\alpha'}(\rho^\ast)\|_1.
\]
\end{theorem}

\begin{proof}
Write
\[
\|\rho^\ast - {\rho^\ast}'\|_1
= \|\Phi_{\lambda,\alpha}(\rho^\ast) - \Phi_{\lambda',\alpha'}({\rho^\ast}')\|_1
\]
and add/subtract $\Phi_{\lambda',\alpha'}(\rho^\ast)$ to obtain
\[
\le \|\Phi_{\lambda,\alpha}(\rho^\ast) - \Phi_{\lambda',\alpha'}(\rho^\ast)\|_1
  + \|\Phi_{\lambda',\alpha'}(\rho^\ast) - \Phi_{\lambda',\alpha'}({\rho^\ast}')\|_1.
\]
The second term is bounded by $k\|\rho^\ast - {\rho^\ast}'\|_1$, yielding
\[
(1-k)\|\rho^\ast - {\rho^\ast}'\|_1
\le \|\Delta\Phi(\rho^\ast)\|_1.
\]
Divide through by $(1-k)$.
\end{proof}

This quantifies how regulator choices affect the equilibrium.

